%%\usepackage[nolist]{acronym}
%\usepackage[acronym,automake,toc,nonumberlist,seeautonumberlist]{glossaries-extra}
%\usepackage[acronym,automake,toc, indexonlyfirst]{glossaries-extra}
%\renewcommand{\descriptionlabel}[1]{\hspace*{\labelsep}\textls[80]{\textbf{\scshape{#1}}}}
\usepackage[acronym,automake,nonumberlist]{glossaries-extra}
\usepackage{glossaries-prefix}
%% Define glossary style
\newglossarystyle{mylist}{%
\setglossarystyle{list}% base this style on the list style
\renewcommand*{\glossentry}[2]{%
\item[\glsentryitem{##1}%
\glstarget{##1}{\glossentryname{##1}}]
\glossentrydesc{##1}\glspostdescription}}
%\glossentrydesc{##1}\glspostdescription\space See first p.\space ##2}}

%% Set glossary style
\setglossarystyle{mylist}
%% Create symbol glossary
\newglossary[slg]{symbolslist}{syi}{syg}{Symbolslist}
\glsaddkey{unit}{\glsentrytext{\glslabel}}{\glsentryunit}{\GLsentryunit}{\glsunit}{\Glsunit}{\GLSunit}
\newglossarystyle{symbunitlong}{%
\setglossarystyle{long3col}% base this style on the list style
\renewenvironment{theglossary}{% Change the table type --> 3 columns
  \begin{longtable}{lp{0.6\glsdescwidth}>{\centering\arraybackslash}p{2cm}}}%
  {\end{longtable}}%
%
\renewcommand*{\glossaryheader}{%  Change the table header
  \bfseries Sign & \bfseries Description & \bfseries Unit \\
  \hline
  \endhead}
\renewcommand*{\glossentry}[2]{%  Change the displayed items
\glstarget{##1}{\glossentryname{##1}} %
& \glossentrydesc{##1}% Description
& \glsunit{##1}  \tabularnewline
}
}
\newglossary[flg]{fake}{fls}{flo}{Fake Entries}
%% Make glossary
\makeglossaries

%%Define acro macro
\newcommand\acro[4][\acronymtype]{
    \newglossaryentry{#2glo}{name={#2}, description={#4.}}
    \newglossaryentry{#2}{type=#1, name={#2},
    description={#3.
%    \glsseeformat[See glossary:]{#2glo}
    }, text={#2}, first={#3
    (#2)\glsadd{#2glo}}, plural={#2s}, firstplural={#3s
    (#2s)\glsadd{#2glo}}, long={#3}, longplural={#3s}}
}
\newcommand\acroAlias[3]{%Define an acronym, used to call the gls but always use the full form
    \newglossaryentry{#1}{name={#2}, description={#3.}}
}
%%Define acro macro
\newcommand\acroNoDef[3][\acronymtype]{
    \newglossaryentry{#2}{type=#1, name={#2},
    description={#3.}, text={#2}, first={#3 (#2)}, plural={#2s}, firstplural={#3s
    (#2s)}, long={#3}, longplural={#3s}}
}

%%Define acro for irregular plurials macro
\newcommand\acroPl[6][\acronymtype]{
    %#1 short sing
    %#2 long sing
    %#3 short pl
    %#4 long pl
    %#5 description
    \newglossaryentry{#2glo}{name={#2}, description={#6.}}
    \newglossaryentry{#2}{type=#1, name={#2},
    description={#3.
%    \glsseeformat[See glossary:]{#2glo}
    }, text={#2}, first={#3
    (#2)\glsadd{#2glo}}, plural={#4}, firstplural={#5 (#4)\glsadd{#2glo}}}
}
%%Define symbol macro
\newcommand\acroSymb[3]{%#1 symbol name, #2 description, #3 units
    \newglossaryentry{#1}{type=symbolslist, name={\ensuremath{#1}},
    description={#2.}, text={#2}, unit={#3},}
}
%(#1s)\glsadd{#1glo}}, see=[Glossary:]{#1glo}}}
\newcommand\acroAlwaysShort[2]{
    %#1 short name
    %#2 long name
    \newglossaryentry{#1}{type=fake, name={#1}, description={#2}, text={#1}, first={#1}, plural={#1s}, firstplural={#1s}}
}
\newcommand\acroAlwaysShortWithDef[3]{
    %#1 short name
    %#2 long name
    %#3 description
    \newglossaryentry{#1glo}{name={#1}, description={#3}}
    \newglossaryentry{#1}{type=fake, name={#1}, description={#2. \glsseeformat[See glossary:]{#1glo}}, text={#1}, first={#1\glsadd{#1glo}}, plural={#1s}, firstplural={#1s\glsadd{#1glo}}}
}
\newcommand\acroAlwaysShortNoPlurialWithDef[3]{
    %#1 short name
    %#2 long name
    %#3 description
    \newglossaryentry{#1glo}{name={#1}, description={#3}}
    \newglossaryentry{#1}{type=fake, name={#1}, description={#2. \glsseeformat[See glossary:]{#1glo}}, text={#1}, first={#1\glsadd{#1glo}}, plural={#1}, firstplural={#1\glsadd{#1glo}}}
}
\newcommand\acroShortSurname[2]{\newglossaryentry{#1}{type=fake, name={#2}, description={#2}, text={#2}, first={#2}, plural={#2s}, firstplural={#2s}}}
%%%
\newcommand\ac[1]{\gls{#1}}
\newcommand\acp[1]{\glspl{#1}}
\newcommand\acs[1]{\glsname{#1}}
\newcommand\acl[1]{\glsentrylong{#1}}

%%symbols
%\acroSymb{BSymb}{Average available bandwidth}{\si{Mbit\cdot s^{-1}}}
%%acronyms
\acroNoDef{2D}{two Dimensional}
\acroNoDef{3D}{Three Dimensional}
\acroNoDef{AR}{Augmented Reality}
\acro{ABR}{Adaptive Bit-Rate}{Familly of adaptive streaming protocols which can
ajust the bit-rate of the downloaded video stream to the measured available downloading
bandwidth. Usually the video is available at  multiple bit-rate on the server
and the client decides which representation of the video to download.}
\acroNoDef{AI}{Artificial Intelligence}
\acroNoDef{API}{Application Programming Interface}
\acroNoDef{AQM}{Active Queue Management}
\acro{AOMedia}{Alliance for Open Media}{The Alliance for Open Media s a Joint
Development Foundation Projects. The goal of this non-profit industry consortium
is to develop openn royalty-free standard for multimedia delivery over the
Internet.}
\acro{AVC}{Advanced Video Coding}{Video codec, also denoted as H.264 or
MPEG-4 Part 10, released in September \num{2004}. Even if AVC can support up to
\gls{8K} resolution videos, it was mainly designed to compress videos video with
resolution up to full HD (FHD).}
\acroNoDef{B}{bidirectional-predicted}
\acroNoDef{CDF}{Cumulative Density Function}
\acroNoDef{CDN}{Content Delivery Network}
\acroNoDef{CMSE}{Commulative Mean Square Error}
\acroNoDef{CMT-SCTP}{Concurrent Multipath Transfer SCTP}
\acroNoDef{CPU}{Central Processing Unit}
\acroNoDef{CWND}{Congestion Window}
\acro{DASH}{Dynamic Adaptive Streaming over HTTP}{TODO}
\acroNoDef{DoF}{Degrees of Freedom}
\acroNoDef{E2E}{End-to-end}
\acroNoDef{EPFL}{Ecole Polytechnique F\'{e}d\'{e}rale de Lausanne}
\acroNoDef{FEC}{Forward Error Correction}
\acroNoDef{FIFO}{First-In First-Out}
\acro{FoV}{Field of View}{Angular size of the \gls{viewport}, often expressed as the vertical angular distance and the horizontal angular distance of the \gls{viewport}}
\acroAlias{VAS}{viewport-adaptive streaming}{An extension of \gls{DASH} that takes into account the user's head orientation prediction, in addition to the
prediction of the available bandwidth, to download video segments that maximize the quality in the displayed part of the omnidirectional video}
\acroNoDef{MAE}{Mean Absolute Error}
\acroNoDef{GOP}{Group of Picture}
\acroNoDef{GPS}{Global Positioning System}
\acroNoDef{GPU}{graphics processing unit}
\acroNoDef{GUI}{Graphical User Interface}
\acroNoDef{HAS}{\ac{HTTP} Adaptive Streaming}
\acroNoDef{HLS}{\gls{HTTP} Live Streaming}
\acroNoDef{HDK2}{Hacker Development Kit 2}
\acroNoDef{HDTV}{high definition television}
\acroNoDef{P2P}{Peer-to-Peer}
\acroNoDef{TV}{television}
\acroNoDef{LF}{Light Field}
\acroNoDef{HEVC}{High Efficiency Video Coding}
\acroNoDef{HM-15}{\ac{HEVC} Test Model - 15}
\acroNoDef{HMD}{Head-Mounted Display}
\acroNoDef{HDS}{\gls{HTTP} Dynamic Streaming}
\acroAlwaysShort{HTTP}{Hypertext Transfer Protocol}
\acroAlwaysShort{HTTP/2}{Hypertext Transfer Protocol version 2}
\acroNoDef{I}{intra-predicted}
\acroAlwaysShort{IETF}{Internet Engineering Task Force}
\acroNoDef{ILP}{Integer Linear Program}
\acroAlwaysShort{IS-PSNR}{}
\acroNoDef{MMORPG}{Massively Multi-user Online Role Playing Games}
\acroNoDef{ISO}{International Organization for Standardization}
\acroNoDef{ISOBMFF}{\gls{ISO} base media file format}
\acroNoDef{JVET}{Joint Video Exploration Team}
\acroNoDef{LAN}{Local Area Network}
\acroNoDef{MAC}{Media Access Control}
\acroNoDef{MCTS}{motion-constrained tile sets}
\acroNoDef{MILP}{Mixed Integer Linear Programming}
\acroNoDef{MOS}{Mean Opinion Score}
\acroNoDef{MPD}{Media Presentation Description}
\acroNoDef{MPEG}{Moving Picture Experts Group}
\acroNoDef{JPEG}{Joint Photographic Experts Group}
\acroNoDef{JPEG 2000}{\acl{JPEG} \num{2000}}
\acroNoDef{3GPP}{$3^{\text{rd}}$ Generation Partnership Project}
\acroNoDef{MPEG-I}{\gls{MPEG} -- Immersive }
\acroNoDef{MP4}{\gls{MPEG}-4 Part 14}
\acroNoDef{MKV}{Matroska}
\acroNoDef{PC}{Point Cloud}
\acroNoDef{ML}{Machine Learning}
\acroNoDef{MPRTP}{Multi-Path Real-time Transport Protocol}
\acroNoDef{MPTCP}{Multi-Path Transmission Control Protocol}
\acroNoDef{MS-SSIM}{Multiscale - Structural Similarity}
\acroNoDef{SSIM}{Structural Similarity}
\acroNoDef{MSS}{Microsoft Smooth Streaming}
\acroNoDef{MSE}{Mean Square Error}
\acroNoDef{MTU}{Maximum Transmission Unit}
\acroNoDef{MVD}{multi-view-plus-depth}
\acroNoDef{MVP}{Multi-ViewPoint}
\acroNoDef{Mbps}{mega bits per second}
\acroNoDef{PCC}{Point Cloud Compression}
\acroNoDef{NTP}{Network Time Protocol}
\acroNoDef{OLIA}{Opportunistic Linked-Increases Algorithm}
\acroNoDef{OMAF}{Omnidirectional MediA Format}
\acroNoDef{OS}{Operating System}
\acroNoDef{OSVR}{Open-Source Virtual Reality}
\acroNoDef{OTT}{Over-The-Top}
\acroNoDef{OpenGL}{Open Graphics Library}
\acroNoDef{P}{inter-predicted}
\acroNoDef{PDF}{probability density function}
\acroNoDef{PMU}{Performance Monitoring Unit}
\acroNoDef{POC}{picture order counts}
\acroNoDef{PSNR}{Peak Signal to Noise Ratio}
\acroNoDef{QEC}{Quality Emphasis Center}
\acro{QER}{Quality Emphasized Region}{Set of viewing direction with higher quality than the rest of the video}
\acroNoDef{QP}{Quantization Parameter}
\acroNoDef{QUIC}{Quick UDP Internet Connections}
\acro{QoE}{Quality of Experience}{Subjective quality experienced by a user viewing a media}
\acroNoDef{RAP}{Random Access Point}
\acroNoDef{LTE}{Long Term Evolution}
\acroNoDef{RPS}{reference picture set}
\acroNoDef{RTCP}{Real-time Transport Control Protocol}
\acroNoDef{RTP}{Real-time Transport Protocol}
\acroNoDef{RTMP}{Real-Time Message Protocol}
\acroNoDef{RTT}{Round-Trip Time}
\acroNoDef{S-PSNR}{Spherical - Peak Signal Noise to Ratio}
\acroNoDef{SCTP}{Stream Control Transmission Protocol}
\acroNoDef{SHVC}{HEVC Scalable Extension}
\acroNoDef{SLERP}{spherical linear interpolation}
\acroNoDef{SRD}{Spatial Relationship Description}
\acroNoDef{SVC}{Scalable Video Coding}
\acroNoDef{TAPS}{Transport Services}
\acroAlwaysShort{TCP}{Transmission Control Protocol}
\acroAlwaysShort{UDP}{User Datagram Protocol}
\acroAlwaysShortNoPlurialWithDef{fps}{Frame per Second}{Number of frame displayed per second. It is sometime denoted as the temporal resolution of the video}
\acroNoDef{UGC}{User-Generated Content}
\acroNoDef{VM}{Virtual Machine}
\acroNoDef{VPN}{Virtual Private Network}
\acroNoDef{VMAF}{Video Multi-Method Assessment Fusion}
\acroNoDef{VQM}{Video Quality Metric}
\acroNoDef{VQMT}{Video Quality Measurement Tool}
\acroNoDef{VQEG}{Video Quality Experts Group}
\acroNoDef{VR}{Virtual Reality}
\acroNoDef{FAMC}{Frame-Based Animated Mesh Compression}
\acroNoDef{VoD}{Video on Demand}
\acroNoDef{URL}{Uniform Resource Locator}
\acroNoDef{RGB-D}{color-depth}
\acroNoDef{CGI}{Computer-Generated Imagery}
\acroNoDef{SVO}{Sparse Voxel Octree}
\acroNoDef{WS-PSNR}{Weighted Spherical - Peak Signal Noise to Ratio}
\acroNoDef{sRTT}{smoothed Round-Trip Time}
\acroNoDef{SLAM}{Simultaneous Localization And Mapping}
\acroNoDef{glTF}{GL Transmission Format}
\acro{4K}{\num{3840} $\times$ \num{2160} pixels}{Spatial resolution of a frame
of a video. The $4$K resolution is also denoted as Ultra-high-definition}
\acroNoDef{8K}{\num{7680} $\times$ \num{4320} pixels}
\acroNoDef{12K}{\num{11520} $\times$ \num{6480} pixels}
\acroNoDef{16K}{\num{15360} $\times$ \num{8640} pixels}
\acroPl{1DoF}{one Degree of Freedom}{1DoF}{one Degree of Freedom}{TODO}
\acroPl{3DoF}{three Degrees of Freedom}{3DoF}{three Degrees of Freedom}{TODO}
\acroPl{6DoF}{six Degrees of Freedom}{6DoF}{six Degrees of Freedom}{TODO}
\acroPl{RoI}{Region of Interest}{RoI}{Region of Interest}{TODO}
%%shortSurnames
\acroShortSurname{RGB}{RGB}
\acroShortSurname{Wi-Fi}{Wi-Fi}
\acroShortSurname{4G}{4G}
\acroShortSurname{YUV}{YUV}
%%raw newglossaryentry
\newglossaryentry{SPSNR}{type=\acronymtype, name={S-PSNR}, text={S-PSNR}, first={Spherical Peak Signal to Noise Ratio~(S-PSNR)}, description={Spherical Peak Signal to Noise Ratio}}
\newglossaryentry{interactive video}{name={interactive video}, description={Video that can be modified while being displayed in response to user's feedbacks.}}
\newglossaryentry{p}{type=\acronymtype, name={p}, description={p}, text={~pixel}, first={~pixel (p)}, plural={p}, firstplural={~pixels (p)}}
\newglossaryentry{s}{type=\acronymtype, name={s}, description={s}, text={second}, first={~second~(s)}, plural={s}, firstplural={~seconds~(s)}}
\newglossaryentry{viewport}{name={viewport}, description={Portion of a 360-degree video displayed to the user.}}
%%%%%TODO: To be removed latter -> multiview
\newglossaryentry{multiview}{type=fake, name={multiview}, description={multi-viewpoint}, text={multi-viewpoint}, first={multi-viewpoint~(MVP)}, plural={(MVPs)}, firstplural={multi-viewpoints~(MVPs)}}
%\newglossaryentry{viewpoint}{type=\acronymtype, name={viewpoint}, description={viewpoint}, text={viewpoint}, first={viewpoint~(VP)}, plural={(VPs)}, firstplural={viewpoints~(VPs)}}
%\newglossaryentry{viewpoint}{type=\acronymtype, name={viewpoint}, description={viewpoint}, text={VP}, first={viewpoint~(VP)}, plural={VPs}, firstplural={viewpoints~(VPs)}}
\newglossaryentry{viewpoint}{type=fake, name={viewpoint}, description={viewpoint}, text={viewpoint}, first={viewpoint}, plural={viewpoints}, firstplural={viewpoints}}
\newglossaryentry{service provider}{name={service provider}, description={Platform ables to stream videos on demand.}}
\newglossaryentry{omnidirectional video}{name={omnidirectional video}, description={Video with pixel in every direction of space. Also known as \SI{360}{\degree} video or spherical video.}}


%%Glossary
\newglossaryentry{asset}{prefix={an\ },name={asset}, first={\textbf{media asset} (in short \textbf{asset})}, firstplural={\textbf{media assets} (in short \textbf{assets})}, description={Any media object that can be embedded into the real world in a specific position}}

\newglossaryentry{slice}{prefix ={an\ },name={alternate reality}, first={\textbf{alternate reality}}, plural={alternate realities}, firstplural={\textbf{alternate realities}}, description={A virtual world that superposes the real world with synthesized content}}

\newglossaryentry{designer}{prefixfirst={an\ },name={designer}, first={\textbf{asset designer} (in short \textbf{designer})}, firstplural={\textbf{\gls{asset} designers} (in short \textbf{designers})}, description={A creator of \glspl{asset}}}

\newglossaryentry{creator}{name={creator}, first={\textbf{\gls{AR} app creator} (in short \textbf{creator})}, firstplural={\textbf{\gls{AR} app creators} (in short \textbf{creators})}, description={A creator of \glspl{AR} app}}

\newglossaryentry{cadaster}{name={cadaster}, first={\textbf{cadaster}}, firstplural={\textbf{cadasters} }, description={A comprehensive virtual land recording of the positions of all \glspl{asset} in a given \gls{slice}.}}

\newglossaryentry{user}{prefix={an\ },name={end-user}, first={\textbf{end-user}}, firstplural={\textbf{end-users}}, description={The perosn who consumes the \gls{AR} app}}

\newglossaryentry{scene}{name={scene}, first={\textbf{scene}}, firstplural={\textbf{scenes} }, description={A portion of the real world on top of which a \gls{slice} superposes some \glspl{asset}}}

\newglossaryentry{spot}{name={spot}, first={\textbf{spot}}, firstplural={\textbf{spots} }, description={A position in the \gls{scene} where \pgls{asset} can be deployed.}}


\acroPl{LoD}{Level of Details}{LoD}{Levels of Details}{TODO}


